%&program=xelatex
%&encoding=UTF-8 Unicode

\documentclass[12pt,letterpaper,twoside,openany,showidx]{book}
\usepackage[silent]{fontspec}
\usepackage{xltxtra}
\usepackage{polyglossia}
\setdefaultlanguage{greek}
\newICUfeature{Contextual}{on}{+calt}
\defaultfontfeatures{Mapping=tex-text,Script=Greek,Contextual=on}
\newcommand{\hlig}[1]{{\addfontfeature{Ligatures=Historic}{#1}}}
\newcommand{\salt}[1]{{\addfontfeature{Style=Alternate}{#1}}}
\setromanfont{Junicode}
\begin{document}
\noindent\Large ΜΗ̃ΝΙΝ ἄ\hlig{ει}δε, ΘΕᾺ, Πηληϊάδεω ἈΧΙΛΗ̃ΟΣ\\
Οὐλομένην, ἣ μυρί᾽ Ἀχαιοῖς ἄλγε᾽ ἔθηκε·\\
Πολλὰς δ᾽ ἰφθίμ\hlig{ου}ς ψυχὰς ἄϊδι προΐαψεν\\
Ἡρώων, αὐτ\hlig{οὺ}ς δ᾽ ἑλώρια τεῦχε κύνεσσιν\\
Οἰωνοῖσί τε πᾶσι· Διὸς δ᾽ ἐτελ\hlig{εί}ε\salt{τ}ο β\hlig{ου}λή·\\
Ἐξ \hlig{οὗ} δὴ τὰ πρῶτα δια\hlig{στ}ήτην ἐρίσαν\salt{τ}ε\\
Ἀτρ\hlig{εΐδ}ης τε, ἄναξ ἀνδρῶν ϗ δῖος Ἀχιλλεύς.\\
    Τίς τ᾽ ἄρ σφωε \salt{θ}εῶν ἔριδι ξυνέηκε μάχε\hlig{σθ}αι;\\
Λητ\hlig{οῦ}ς καὶ Διὸς υἱός· ὃ γὰρ βασιλῆϊ χολωθ\hlig{εὶ}ς\\
Ν\hlig{οῦ}σον ἀνὰ \hlig{στρ}ατὸν ὄρσε κακήν· ὀλέκον\salt{τ}ο δὲ λαοί·\\
Οὕνεκα τὸν Χρύσην ἠτίμησ᾽ ἀρητῆρα\\
Ἀτρ\hlig{εΐδ}ης· ὃ γὰρ ἦλθε \salt{θ}οὰς ἐπὶ νῆας Ἀχαιῶν\\
Λυσόμενός τε \salt{θ}ύγα\salt{τ}ρα φέρων τ᾽ ἀπερ\hlig{εί}σι᾽ ἄποινα,\\
Στέμμα\salt{τ᾽} ἔχων ἐν χερσὶν ἑκη\salt{β}όλ\hlig{ου} Ἀπόλλωνος,\\
Χρυσέῳ ἀνὰ σκήπ\salt{τ}ρῳ καὶ ἐλίσσε\salt{τ}ο πάν\salt{τ}ας Ἀχ\hlig{αιού}ς,\\
Ἀτρ\hlig{εΐδ}α δὲ μάλι\hlig{στα} δύω, κοσμήτορε λαῶν·\\
    Ἀτρ\hlig{εΐδ}αι τε, καὶ ἄλλοι ἐϋκνήμιδες Ἀχαιοί,\\
Ὑμῖν μὲν \salt{θ}εοὶ δοῖεν Ὀλύμπια δώμα\salt{τ᾽} ἔχοντες\\
Ἐκπέρσαι Πριάμοιο πόλιν, εὖ δ᾽ οἴκαδ᾽ ἱκέ\hlig{σθ}αι:\\
Παῖδα δέ μοὶ λύσαι\salt{τ}ε \salt{φ}ίλην, τὰ δ᾽ ἄποινα δέχε\hlig{σθ}ε,\\
Ἁζόμενοι Διὸς υἱὸν ἑκη\salt{β}όλον Ἀπόλλωνα.\\
    Ἔνθ᾽ ἄλλοι μὲν πάντες ἐπευφήμησαν Ἀχαιοὶ,\\
Αἰδ\hlig{εῖσθ}αί \salt{θ}᾽ ἱερῆα, ϗ ἀγλαὰ δέχθαι ἄποινα·\\
Ἀλλ᾽ \hlig{οὐ}κ Ἀτρ\hlig{εΐδ}ῃ Ἀγαμέμνονι ἥνδανε \salt{θ}υμῷ,\\
Ἀλλὰ κακῶς ἀφί\hlig{ει}, κρα\salt{τ}ερὸν δ᾽ ἐπὶ μῦθον ἔτελλε·\\
    Μή σε, γέρον κοίλῃσιν ἐγὼ παρὰ νηυσὶ κιχ\hlig{εί}ω\\
Ἢ νῦν δηθύνον\salt{τ᾽} ἢ ὕ\hlig{στε}ρον αὖτις ἰόν\salt{τ}α,\\
Μή νύ τοι \hlig{οὐ} \hlig{χραί}σμῃ σκῆπ\salt{τ}ρον ϗ \hlig{στέ}μμα \salt{θ}εοῖο.\\
Τὴν δ᾽ ἐγὼ \hlig{οὐ} λύσω, πρίν μιν ϗ γῆρας ἔπ\hlig{ει}σιν,\\
Ἡμετέρῳ ἐνὶ οἴκῳ ἐν Ἄργεϊ τηλόθι πάτρης\\
Ἱ\hlig{στὸ}ν ἐποιχομένην, ϗ ἐμὸν λέχος ἀν\salt{τ}ιόωσαν·\\
Ἀλλ᾽ ἴθι μή μ᾽ ἐρέθιζε σαώτερος ὥς κε νέη\hlig{αι}.\\
    Ὣς ἔφα\salt{τ᾽}· ἔδδ\hlig{ει}σεν δ᾽ ὃ γέρων, ϗ ἐπ\hlig{εί}θε\salt{τ}ο μύθῳ·\\
Βῆ δ᾽ ἀκέων παρὰ \salt{θ}ῖνα πολυφλοίσ\salt{β}οιο \salt{θ}αλάσσης·\\
Πολλὰ δ᾽ ἔπ\hlig{ει}\salt{τ᾽} ἀπάνευθε κιὼν ἠρᾶθ᾽ ὃ γερ\hlig{αι}ὸς\\
Ἀπόλλωνι ἄνακ\salt{τ}ι, τὸν ἠΰκομος τέκε Λητώ·\\
    Κλῦθί μευ Ἀργυρότοξ᾽, ὃς Χρύσην ἀμφι\salt{βέβ}ηκας\\
Κίλλάν τε ζαθέην Τενέδοιό τε ἶφι ἀνάσσ\hlig{ει}ς,\\
Σμινθεῦ \hlig{εἴ} πο\salt{τ}έ τοι χαρίεν\salt{τ᾽} ἐπὶ νηὸν ἔρεψα,\\
Ἢ \hlig{εἰ} δή πο\salt{τ}έ τοι κατὰ πίονα μηρί᾽ ἔκηα\\
Ταύρων ἠδ᾽ \hlig{αἰ}γῶν, τὸ δέ μοι κρήηνον ἐέλδωρ·\\
Τίσ\hlig{ει}αν Δαναοὶ ἐμὰ δάκρυα σοῖσι βέλεσσιν.\\
    Ὣς ἔφα\salt{τ᾽} εὐχόμενος· τ\hlig{οῦ} δ᾽ ἔκλυε Φο\salt{ῖβ}ος Ἀπόλλων,\\
Βῆ δὲ κα\salt{τ᾽} \hlig{oὐ}λύμποιο καρήνων χωόμενος κῆρ,\\
Τόξ᾽ ὤμοισιν ἔχων ἀμφηρεφέα τε φαρέτρην·\\
Ἔκλαγξαν δ᾽ ἄρ᾽ ὀϊ\hlig{στο}ὶ ἐπ᾽ ὤμων χωομένοιο,\\
Αὐτ\hlig{οῦ} κινηθέν\salt{τ}ος· ὃ δ᾽ ἤϊε νυκ\salt{τ}ὶ ἐοικώς.\\
Ἕζε\salt{τ᾽} ἔπ\hlig{ει}\salt{τ᾽} ἀπάνευθε νεῶν, με\salt{τὰ} δ᾽ ἰὸν ἕηκε·\\
Δ\hlig{ει}νὴ δὲ κλα\hlig{γγὴ} γένε\salt{τ᾽} ἀργυρέοιο βιοῖο.\\
Οὐρῆας μὲν πρῶτον ἐπῴχε\salt{τ}ο ϗ κύνας ἀργ\hlig{ού}ς,\\
Αὐτὰρ ἔπ\hlig{ει}\salt{τ᾽} \hlig{αὐ}τοῖσι βέλος ἐχεπευκὲς ἐφι\hlig{εὶ}ς\\
Βάλλ᾽· \hlig{αἰεὶ} δὲ πυρ\hlig{αὶ} νεκύων κ\hlig{αί}ον\salt{τ}ο \salt{θ}αμ\hlig{ειαί}.\\
Ἐννῆμαρ μὲν ἀνὰ \hlig{στρ}ατὸν ᾤχετο κῆλα \salt{θ}εοῖο,\\
Τῇ δεκάτῃ δ᾽ ἀγορὴν δὲ καλέσσα\salt{τ}ο λαὸν Ἀχιλλεύς·\\
Τῷ γὰρ ἐπὶ φρεσὶ \salt{θ}ῆκε \salt{θ}εὰ λευκώλενος Ἥρη·\\
Κήδε\salt{τ}ο γὰρ Δαναῶν, ὅτι ῥα \salt{θ}νήσκον\salt{τ}ας ὁρᾶτο.\\
Οἳ δ᾽ ἐπ\hlig{εὶ οὖ}ν ἤγερθεν, ὁμηγερέες τ᾽ ἐγένον\salt{τ}ο,\\
Τοῖσι δ᾽ ἀνι\hlig{στά}μενος μετέφη πόδας ὠκὺς Ἀχιλλεύς·\\
    Ἀτρ\hlig{εΐδ}η νῦν ἄμμε παλιμπλα\salt{γ}χθέντας ὀΐω\\
Ἂψ ἀπονο\hlig{στή}σ\hlig{ειν, εἴ} κεν \salt{θ}άνα\salt{τ}όν γε φύγοιμεν·\\
Εἰ δὴ ὁμ\hlig{οῦ} πόλεμός τε δαμᾷ ϗ λοιμὸς Ἀχ\hlig{αιού}ς·\\
Ἀλλ᾽ ἄγε δή τινα μάντιν ἐρ\hlig{εί}ομεν, ἢ ἱερῆα,\\
Ἢ κ\hlig{αὶ} ὀν\hlig{ει}ροπόλον, κ\hlig{αὶ} γάρ τ᾽ ὄναρ ἐκ Διός ἐ\hlig{στ}ιν,\\
Ὅς κ᾽ \hlig{εἴ}ποι ὅ τι τόσσον ἐχώσα\salt{τ}ο Φο\salt{ῖβ}ος Ἀπόλλων·\\
Εἴτ᾽ ἄρ᾽ ὅ γ᾽ εὐχωλῆς ἐπιμέμφετ\hlig{αι} ἠδ᾽ ἑκατόμ\salt{β}ης·\\
Αἴ κέν πως ἀρνῶν κνίσσης \hlig{αἰ}γῶν τε τελ\hlig{εί}ων\\
Β\hlig{ού}λε\salt{τ}\hlig{αι} ἀντιάσας ἡμῖν ἀπὸ λοιγὸν ἀμῦν\hlig{αι}.
\end{document}
